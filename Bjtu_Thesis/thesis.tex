% 导言部分
\documentclass[zihao = -4]{ctexbook}  % 定义文档类型是一本书,以使毕业论文可以分章节
\usepackage[backend=biber, style=gb7714-2015]{biblatex}  % 设置参考文献样式及引用样式,TexLive中已经有GB7714-2015样式
\usepackage[bookmarksnumbered]{hyperref}  % 添加超链接,bookmarksnumbered给书签编号
\addbibresource{references.bib}  % 导入参考文献bib文件
\usepackage{fontspec}  % 导入设置默认字体的宏包
\setmainfont{Times New Roman}  % 设置默认字体为Times New Roman
\linespread{1.4}  % 设置默认1.4倍行距
\setlength{\parskip}{0pt}  % 设置默认段落间距为0
\usepackage{amsmath}  % 导入数学公式宏包
\usepackage[a4paper, twoside, left=35mm, right=25mm, top=30mm, bottom=25mm, headheight=16mm, footskip=10mm]{geometry}  % 设置页面布局,装订线10mm被算在左边距里
\usepackage{fancyhdr} \pagestyle{fancy}  % 导入页眉页脚包,并使用fancy样式
\fancyhf{}\fancyhead[EC]{\zihao{4}北京交通大学博士学位论文} \fancyhead[OC]{\zihao{4}\leftmark}  % 按要求更改页眉内容
\fancyfoot[C]{\zihao{-5}\thepage}  % 在页脚处加入页码
\fancypagestyle{plain}{
	\fancyhf{}
	\fancyhead[EC]{\zihao{4}北京交通大学博士学位论文}
	\fancyhead[OC]{\zihao{4}\leftmark}
	\fancyfoot[C]{\zihao{-5}\thepage}
}  % 更改每一chapter开头使用的plain样式

\ctexset{  % 设置章节格式
	chapter/name = {,},
	chapter/number = \arabic{chapter},  % 章序号为Arabic numbers
	chapter/format = \centering \zihao{3} \heiti \linespread{1},
	chapter/beforeskip = {4.8pt},  % 要求为24pt,24pt-16pt*1.2 = 4.8pt 段前间距
	chapter/afterskip = {18pt},	 % 段后间距 18pt
	section/format = \raggedright \zihao{-3} \heiti ,  % 左对齐,小三号,黑体
	section/beforeskip = {6pt},  % 要求为24pt, 24pt-15pi*1.2 = 6pt
	section/afterskip = {18pt},
	subsection/format = \raggedright \zihao{4} \heiti ,
	subsection/beforeskip = {7.2pt},  % 要求为24pt, 24pt-14pi*1.2 = 7.2pt
	subsection/afterskip = {18pt}  % 段后间距
}
\usepackage{graphicx}  % 导入图像宏包
\graphicspath{{figures/}}  % 图片路径
\usepackage{float}


% 文档部分
\begin{document}
	% !TeX spellcheck = en_GB
% 开头的封面1
\thispagestyle{empty}  % 去掉当前页的页眉页脚
{\fontsize{10}{10}\selectfont \vspace*{1em}}
{\fontsize{42}{22}\selectfont \vspace{1em}}
\centerline{\includegraphics[height=2.77cm, width=10.71cm]{xiaominglogo.eps}}
{\fontsize{18}{18}\selectfont \vspace{1em}}
{\fontsize{22}{22}\selectfont \centerline{博士学位论文}
{\fontsize{22}{22}\selectfont \vspace{2em}}
{\fontsize{15}{15}\selectfont \centerline{论文标题}}
{\fontsize{18}{18}\selectfont \vspace{1em}}
{\fontsize{15}{15}\selectfont \centerline{The title of dissertation}}
{\fontsize{18}{18}\selectfont \vspace{5em}}
{\fontsize{14}{14}\selectfont
\begin{table}[h]
	\centering
	\begin{tabular}{r l l}
		作者: & 孙悟空 \\
		\\
		导师: & 唐\hspace{12pt}僧 & 教授 \\
	\end{tabular}
\end{table}
{\fontsize{12}{12}\selectfont \vspace*{8em}}
{\fontsize{14}{14}\selectfont \centerline{北京交通大学}}
{\fontsize{15}{15}\selectfont \vspace{1em}}
{\fontsize{14}{14}\selectfont \centerline{\today}}


% 反面空白
\newpage
\thispagestyle{empty}  % 去掉当前页的页眉页脚
{\ }


% 开头的版权使用说明书	
\chapter*{学位论文版权使用说明书}
\thispagestyle{empty}  % 去掉当前页的页眉页脚
{\fontsize{12}{12}\selectfont
本学位论文作者完全了解北京交通大学有关保留、使用学位论文的规定。特授权北京交通大学可以将学位论文的全部或部分内容编入有关数据库进行检索,提供阅览服务,并采用影印、缩印或扫描等复制手段保存、汇编以供查阅和借阅。同意学校向国家有关部门或机构送交论文的复印件和磁盘。学校可以为存在馆际合作关系的兄弟高校用户提供文献传递服务和交换服务。
\par
(保密的学位论文在解密后适用本授权说明)
\vspace{5em}
\begin{table}[h]
	\begin{tabular}{l l}
		学位论文作者签名: & 导师签名: \\
		\\
		签字日期:\hspace{5ex}年\hspace{3ex}月\hspace{3ex}日 \hspace*{12ex}& 签字日期:\hspace{5ex}年\hspace{3ex}月\hspace{3ex}日 \\
	\end{tabular}
\end{table}
}

% 反面空白
\newpage
\thispagestyle{empty}  % 去掉当前页的页眉页脚
{\ }


% 开头的封面2
\newpage
\thispagestyle{empty}  % 去掉当前页的页眉页脚

{\fontsize{10.5}{10.5}\selectfont \centerline{学校代码:10004 \hfill 密级:公开}}
{\fontsize{10.5}{10.5}\selectfont \vspace{50pt}}
{\fontsize{42}{42}\selectfont \centerline{\kaishu 北京交通大学}}
{\fontsize{22}{22}\selectfont \centerline{博士学位论文} \vspace{2em}}
{\fontsize{15}{15}\selectfont
	\centerline{论文标题}
	\vspace{1em}
	\centerline{the title of dissertation}
	\vspace{3em}
}
{\fontsize{14}{14}\selectfont
\begin{table}[H]
	\centering
	\begin{tabular}{lp{3cm}ll}
		作者姓名: & 孙悟空 & 学\hspace{24pt}号: & 13111111\\ \\
		导师姓名: & 唐\hspace{12pt}僧 & 职\hspace{24pt}称: & 教授\\ \\
		学位类别: & 工学 & 学位级别: & 博士\\ \\
		学科专业: & 西游 & 研究方向: & 取经\\	\\	
	\end{tabular}
\end{table}
}
{\fontsize{14}{14}\selectfont \vspace{8em}}
{\fontsize{14}{14}\selectfont \centerline{北京交通大学}  \centerline{\today}}



% 反面空白
\newpage
\thispagestyle{empty}  % 去掉当前页的页眉页脚
{\ }

\chapter*{致谢}
\thispagestyle{empty}  % 去掉当前页的页眉页脚
{\fontsize{12}{12}\selectfont
感谢唐玄藏师傅,感谢大唐国的项目支持、经费支持,以及提供白龙马一匹。	
}


% 反面空白
\newpage
\thispagestyle{empty}  % 去掉当前页的页眉页脚
{\ }  % 包括封面1、学位论文版本使用授权书、封面2、致谢	
	% 前置部分
	\frontmatter

	\include{zhaiYao}  % 包括中文摘要、英文摘要
	
	\tableofcontents  % 生成目录	
	\addcontentsline{toc}{chapter}{目录}  % 手动把“目录”编进目录的chapter标题
	
	\renewcommand{\listfigurename}{插图目录}
	\listoffigures
	\addcontentsline{toc}{chapter}{插图目录}
		
	\renewcommand{\listtablename}{表格目录}
	\listoftables
	\addcontentsline{toc}{chapter}{表格目录}
	
	\chapter{主要符号}
花椒面中坚力量  % 主要符号
	
	% 正文部分
	\mainmatter	
	\include{chapter1}  % 插入第1章内容
	\chapter{晶体结构中的孔隙对胶体晶体剪切模量的影响}
\section{引言}

在一定合适的条件下,悬浮液中的带电胶体颗粒能够自组装成类似于原子晶体的晶体结构 \cite{Luo2010,Russel2003, Anderson2002},称为胶体晶体。
由于晶格长度尺度及响应时间这两个量更加容易得到,胶体晶体已经成为凝聚态物质研究的很有价值的模型体系\cite{Yethiraj2007,Yunker2014}。

对材料弹性性质的研究是认识一些物理机制的新途径\cite{Wette2004,Anderson2013,Meng2014,Kung2004},
这些物理机制包括颗粒间相互作用、材料结构、力学性能及相变规律等。
近年来,对胶体晶体弹性性质已经有了一些实验和理论的研究\cite{Crandall1977,Wette2002,Zhou2013,Ouyang2014,Tang2008}:
实验研究方面,使用了扭转共振光谱法来测量胶体晶体的剪切弹性模量\cite{Wette2003,Zhou2015,Schope2001},
在测量过程中,需要将合适频率的旋转振动作用于胶体晶体上并激发出共振驻波;
理论研究方面,人们把根据单一原子固体的弹性常数与原子间相互作用势之间的关系模型\cite{Fuchs1936a,Fuchs1936,Johnson1972}引入到胶体晶体中,
推导出了胶体晶体的剪切模量与胶体颗粒相互作用势之间的关系模型\cite{Dubois-Violette1980,Joanny1979}。

迄今为止,在剪切模量--相互作用势关系模型中,人们一直使用Yukawa作用势表达式来描述两个带电胶体颗粒之间的相互作用势\cite{Wette2004}。
在此作用势模型表达式中有一个重要的参量---胶体颗粒的表面有效电荷$Z_e$。
故原则上当知道胶体晶体中带电颗粒的表面有效电荷$Z_e$时,就可以通过剪切模量--相互作用势关系模型来计算胶体晶体的剪切模量$G$。
然而,却没有人直接使用此关系模型来确定胶体晶体的剪切模量,
将颗粒表面有效电荷代入到关系模型中计算得到的剪切模量值,要比使用扭转共振光谱实验方法得到的值大很多。
实际上已经有研究\cite{Wette2002}显示,
使用剪切模量--相互作用势关系模量拟合不同体积分数时胶体晶体剪切模量测量值得到的胶体颗粒表面有效电荷值(弹性有效电荷),
总是要比电导率--数密度法得到迁移有效电荷$Z_σ$小40\%左右。

很明显已有的剪切模量--相互作用势关系模量仅适用于没有缺陷的理想晶体(perfect crystal)。但实际情况和原子晶体一样,
胶体晶体中也有各种不同的晶体缺陷(crystallographic defects)存在\cite{Yoshida1991},使晶体的剪切刚度(shear stiffness)降低。
例如\cite{Zhou2011}在胶体结晶过程的BCC-FCC转变过程中,
因为转变状态时结构非平衡导致的晶体缺陷,使胶体晶体的胶体晶体的剪切模量远小于理论估计值。
另一个晶体结构缺陷的典型\cite{Zhou2013,Ouyang2014,Yoshida1991}是由带高表面电荷的胶体颗粒形成的胶体晶体,其内部存在有孔隙结构。
另外也有研究\cite{Ouyang2014,Tata2015}指出,当带高电荷颗粒胶体晶体中有孔隙存在时
,Sogami-Ise(SI)作用势比Yukawa作用势更适合去描述胶体颗粒间的相互作用。

为此,本章将优化对于胶体晶体的现有的剪切模量--相互作用势关系模型,使人们能够根据已知的颗粒表面有效电荷直接计算胶体晶体的剪切模量值。
我们将从两个方面对理论模型进行修改,使其适用于内部含有孔隙的胶体晶体,
并使用扭转共振光谱法测量多组由同种单分散颗粒形成的不同颗粒数密度的胶体晶体体积模量值。
通过理论与实验的结果对比,将发现把迁移有效电荷代入到优化后的关系模型中得到的理论结果与实验得到值一致。

\section{理论模型}

原有的适于没有孔隙理想胶体晶体的剪切模量--相互作用势关系模型,在体心立方(BCC)和面心立方(FCC)结构中的表达式分别为

\begin{equation}
	G_{\text{bcc}}\left(d_\text{bcc}\right)={f_A}\frac{3\sqrt{3}}{4{d_\text{bcc}}^3}
	\left(\frac{4}{9}{d_\text{bcc}}^2\left.\frac{\partial^2V(r)}{\partial r^2}\right|_{r=d_\text{bcc}}
	+\frac{8}{9}d_\text{bcc}\left.\frac{\partial V(r)}{\partial r}\right|_{r=d_\text{bcc}}\right)
\end{equation}
和
\begin{equation}
	G_{\text{fcc}}\left(d_\text{fcc}\right)={f_A}\frac{\sqrt{2}}{{d_\text{fcc}}^3}
\left(\frac{1}{2}{d_\text{fcc}}^2\left.\frac{\partial^2V(r)}{\partial r^2}\right|_{r=d_\text{fcc}}
+\frac{3}{2}d_\text{fcc}\left.\frac{\partial V(r)}{\partial r}\right|_{r=d_\text{fcc}}\right)\text{,}
\end{equation}
式中$d_\text{bcc}$和$d_\text{fcc}$分别为BCC结构和FCC结构时的最近相邻颗粒间距,
$V(r)$是当两个颗粒间距为$r$时的有效相互作用势,


\section{材料与方法}
\section{结果与讨论}
\section{本章小结}  % 插入第2章内容
	\include{chapter3}  % 插入第3章内容
	\include{chapter4}  % 插入第4章内容
	
	\printbibliography  % 在正文后打印参考文献
	\addcontentsline{toc}{chapter}{参考文献}  % 手动将“参考文献”编进目录的chapter标题
	
	
	% 附录部分
	\appendix
	\chapter{}
\vspace{8pt}
\centerline{\zihao{3}{\heiti{\linespread{1}{附录A标题}}}}\vspace{16pt}


{\zihao{5}{\linespread{1.27}{\setlength{\parskip}{0pt}
			
附录A正文

}}}



\chapter{}
\vspace{8pt}
\centerline{\zihao{3}{\heiti{\linespread{1}{附录A标题}}}}
\vspace{16pt}


{\zihao{5}{\linespread{1.27}{\setlength{\parskip}{0pt}
			
			附录A正文
			
}}}

  % 附录,包括大篇幅的代码等

	
	\backmatter
	\chapter{研究成果}
\begin{enumerate}
	\item 发表论文
		\begin{enumerate}
			\item 某某某, \textbf{某某某}, 某某某, 某某某.
			容器内角处流体界面特性与Surface Evolver程序适用性的研究\CJKsetecglue{}[J].  % \CJKsetecglue{}是为了去掉中英文混排时出现的空格
			物理学报, 2012, 61(16): 166801.
			(SCI: 000309089200054)
			
			\item 某某某, 某某某, 某某某, \textbf{某某某}, 某某某.
			微重力条件下不同截面形状管中毛细流动的实验研究\CJKsetecglue{}[J].
			物理学报, 2013, 62(13): 134702.
			(SCI: 000322566300049, EI: 20133216588410)
			
			\item 某某某, \textbf{某某某}, 某某某, 某某某.
			微重力条件下与容器连通的毛细管中的毛细流动研究\CJKsetecglue{}[J].
			物理学报, 2015, 64(12):124703.
			(SCI: 000356793900034, EI: 20153001051157)			
						
			\item \textbf{某某某}, 某某某, 某某某, 某某某, 某某某, 某某某.
			带电胶体粒子弹性有效电荷测量的理论改进\CJKsetecglue[J].
			物理学报. 2017, 66(6): 066102. (SCI, EI)
			
			\item Bauer J, Bauer J, Bauer J, Bauer J, \textbf{Bauer J}, Bauer J.
			A Numerical Study on Metallic Powder Flow in Coaxial Laser Cladding[J].
			Journal of Applied Fluid Mechanics, 2016, 9(5): 2247-2256.
			(SCI: 000383413800016)
			
			\item \textbf{Bauer J}, Bauer J, Bauer J,Bauer J, Bauer J.
			Effect of void structures in crystalline structure on the shear moduli of charged colloidal crystals[J].
			Colloids and Surfaces, A: Physicochemical and Engineering Aspects, 2017, 516: 115-120.
			(SCI, EI: 20165203184747)

			
		\end{enumerate}
	\item 参考科研项目
		\begin{enumerate}
			\item 国家自然科学基金(******): ******
			\item 国家自然科学基金(******): ******
			\item 国家自然科学基金(******): ******
		\end{enumerate}
\end{enumerate}

%{\zihao{5}{\linespread{1.27}{\setlength{\parskip}{0pt}

\chapter{独创性声明}
{\fontsize{10.5}{16}\selectfont
本人声明所呈交的学位论文是本人在导师指导下进行的研究工作和取得的研究成果,除了文中特别加以标注和致谢之处外,论文中不包含其他人已经发表或撰写过的研究成果,也不包含为获得北京交通大学或其他教育机构的学位或证书而使用过的材料。与我一同工作的同志对本研究所做的任何贡献均已在论文中作了明确的说明并表示了谢意。
\vspace{3em}
\par
学位论文作者签名:\hspace{6.5em}签字日期:\hspace{4em}年\hspace{2em}月\hspace{2em}日
}


\chapter{学位论文数据集}
afafafasfasfasfasdf
  % 包括研究成果,独创性声明,学位论文数据集
	
\end{document}