\chapter{晶体结构中的孔隙对胶体晶体剪切模量的影响}
\section{引言}

在一定合适的条件下,悬浮液中的带电胶体颗粒能够自组装成类似于原子晶体的晶体结构 \cite{Luo2010,Russel2003, Anderson2002},称为胶体晶体。
由于晶格长度尺度及响应时间这两个量更加容易得到,胶体晶体已经成为凝聚态物质研究的很有价值的模型体系\cite{Yethiraj2007,Yunker2014}。

对材料弹性性质的研究是认识一些物理机制的新途径\cite{Wette2004,Anderson2013,Meng2014,Kung2004},
这些物理机制包括颗粒间相互作用、材料结构、力学性能及相变规律等。
近年来,对胶体晶体弹性性质已经有了一些实验和理论的研究\cite{Crandall1977,Wette2002,Zhou2013,Ouyang2014,Tang2008}:
实验研究方面,使用了扭转共振光谱法来测量胶体晶体的剪切弹性模量\cite{Wette2003,Zhou2015,Schope2001},
在测量过程中,需要将合适频率的旋转振动作用于胶体晶体上并激发出共振驻波;
理论研究方面,人们把根据单一原子固体的弹性常数与原子间相互作用势之间的关系模型\cite{Fuchs1936a,Fuchs1936,Johnson1972}引入到胶体晶体中,
推导出了胶体晶体的剪切模量与胶体颗粒相互作用势之间的关系模型\cite{Dubois-Violette1980,Joanny1979}。

迄今为止,在剪切模量--相互作用势关系模型中,人们一直使用Yukawa作用势表达式来描述两个带电胶体颗粒之间的相互作用势\cite{Wette2004}。
在此作用势模型表达式中有一个重要的参量---胶体颗粒的表面有效电荷$Z_e$。
故原则上当知道胶体晶体中带电颗粒的表面有效电荷$Z_e$时,就可以通过剪切模量--相互作用势关系模型来计算胶体晶体的剪切模量$G$。
然而,却没有人直接使用此关系模型来确定胶体晶体的剪切模量,
将颗粒表面有效电荷代入到关系模型中计算得到的剪切模量值,要比使用扭转共振光谱实验方法得到的值大很多。
实际上已经有研究\cite{Wette2002}显示,
使用剪切模量--相互作用势关系模量拟合不同体积分数时胶体晶体剪切模量测量值得到的胶体颗粒表面有效电荷值(弹性有效电荷),
总是要比电导率--数密度法得到迁移有效电荷$Z_σ$小40\%左右。

很明显已有的剪切模量--相互作用势关系模量仅适用于没有缺陷的理想晶体(perfect crystal)。但实际情况和原子晶体一样,
胶体晶体中也有各种不同的晶体缺陷(crystallographic defects)存在\cite{Yoshida1991},使晶体的剪切刚度(shear stiffness)降低。
例如\cite{Zhou2011}在胶体结晶过程的BCC-FCC转变过程中,
因为转变状态时结构非平衡导致的晶体缺陷,使胶体晶体的胶体晶体的剪切模量远小于理论估计值。
另一个晶体结构缺陷的典型\cite{Zhou2013,Ouyang2014,Yoshida1991}是由带高表面电荷的胶体颗粒形成的胶体晶体,其内部存在有孔隙结构。
另外也有研究\cite{Ouyang2014,Tata2015}指出,当带高电荷颗粒胶体晶体中有孔隙存在时
,Sogami-Ise(SI)作用势比Yukawa作用势更适合去描述胶体颗粒间的相互作用。

为此,本章将优化对于胶体晶体的现有的剪切模量--相互作用势关系模型,使人们能够根据已知的颗粒表面有效电荷直接计算胶体晶体的剪切模量值。
我们将从两个方面对理论模型进行修改,使其适用于内部含有孔隙的胶体晶体,
并使用扭转共振光谱法测量多组由同种单分散颗粒形成的不同颗粒数密度的胶体晶体体积模量值。
通过理论与实验的结果对比,将发现把迁移有效电荷代入到优化后的关系模型中得到的理论结果与实验得到值一致。

\section{理论模型}

原有的适于没有孔隙理想胶体晶体的剪切模量--相互作用势关系模型,在体心立方(BCC)和面心立方(FCC)结构中的表达式分别为

\begin{equation}
	G_{\text{bcc}}\left(d_\text{bcc}\right)={f_A}\frac{3\sqrt{3}}{4{d_\text{bcc}}^3}
	\left(\frac{4}{9}{d_\text{bcc}}^2\left.\frac{\partial^2V(r)}{\partial r^2}\right|_{r=d_\text{bcc}}
	+\frac{8}{9}d_\text{bcc}\left.\frac{\partial V(r)}{\partial r}\right|_{r=d_\text{bcc}}\right)
\end{equation}
和
\begin{equation}
	G_{\text{fcc}}\left(d_\text{fcc}\right)={f_A}\frac{\sqrt{2}}{{d_\text{fcc}}^3}
\left(\frac{1}{2}{d_\text{fcc}}^2\left.\frac{\partial^2V(r)}{\partial r^2}\right|_{r=d_\text{fcc}}
+\frac{3}{2}d_\text{fcc}\left.\frac{\partial V(r)}{\partial r}\right|_{r=d_\text{fcc}}\right)\text{,}
\end{equation}
式中$d_\text{bcc}$和$d_\text{fcc}$分别为BCC结构和FCC结构时的最近相邻颗粒间距,
$V(r)$是当两个颗粒间距为$r$时的有效相互作用势,


\section{材料与方法}
\section{结果与讨论}
\section{本章小结}