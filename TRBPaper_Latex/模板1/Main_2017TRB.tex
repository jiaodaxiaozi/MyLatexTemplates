%
% Transportation Research Board conference paper template
% version 2.1.1
% 
% David R. Pritchard, http://davidpritchard.org
%   1.0   - Mar. 2009
%   1.1   - Sep. 2011, fixes for captions
%   2.0   - Mar. 2012, Reorganized title page incl. automatic counters
%   2.1   - Jul. 2015, Automatic total word counter and more formattings
%   2.1.1 - Jan. 2016, Minor modifications and first uploaded to Github
%   2.1.1 - May. 2016, created a lite version for people to use directly on TeX without Sweave options

% PAGE LAYOUT
%------------------------------------------

% Custom paper settings...
\documentclass[titlepage,oneside,letterpage,12pt]{article}

\oddsidemargin 0.0in
\topmargin -0.5in
\headheight 0.3in
\headsep 0.2in
\textwidth 6.5in
\textheight 9.0in
\setlength{\parindent}{0.5in}

% PAGE HEADER
%------------------------------------------
% Adjust the header text below (INSERT AUTHORS HERE)
\oddsidemargin 0.0in
\usepackage[tiny,rm]{titlesec}
\newpagestyle{trbstyle}{
\sethead{Yan}{}{\thepage}% Change to the last names of the authors, use et al. if 4+ authors
}
\pagestyle{trbstyle}

% HEADINGS
%------------------------------------------
% Line spacing: 12pt before section titles
\renewcommand*{\refname}{\uppercase{References}}
\titleformat{\section}{\bfseries}{}{0pt}{\uppercase}
\titlespacing*{\section}{0pt}{12pt}{*0}
\titleformat{\subsection}{\bfseries}{}{0pt}{}
\titlespacing*{\subsection}{0pt}{12pt}{*0}
\titleformat{\subsubsection}{\itshape}{}{0pt}{}
\titlespacing*{\subsubsection}{0pt}{12pt}{*0}

% LISTS
%------------------------------------------
% Adjust lists a little. Not quite perfectly fitting TRB style, but vaguely
% close at least.
\usepackage{enumitem}
\setlist[1]{labelindent=0.5in,leftmargin=*}
\setlist[2]{labelindent=0in,leftmargin=*}
\setlist{nosep} % eliminate extra verticle spacings between items

% CAPTIONS
%------------------------------------------
% Get the captions right. Authors must still be careful to use "Title Case"
% for table captions, and "Sentence case." for figure captions.
\usepackage{ccaption}
\usepackage{amsmath}
\makeatletter
\renewcommand{\fnum@figure}{\textbf{FIGURE~\thefigure} }
\renewcommand{\fnum@table}{\textbf{TABLE~\thetable} }
\makeatother
\captiontitlefont{\bfseries \boldmath}
\captiondelim{\;}
%\precaption{\boldmath}
%\newcommand{\reffig}[1]{Figure \ref{#1}}

% FONTS
%------------------------------------------
% Times for text and math
\usepackage{mathptmx}

% Some pdf conversion tricks? Unsure.
\usepackage[T1]{fontenc}
\usepackage{textcomp}
% Fonts will be broken by Sweave without this option
%\usepackage[noae]{Sweave}


% CITATIONS
%------------------------------------------
% TRB uses an Author (num) citation style. I haven't found a way to make
% LaTeX/Bibtex do this automatically using the standard \cite macro, but
% this modified \trbcite macro does the trick.

% TODO: sort&compress option?
\usepackage[sort&compress,numbers]{natbib}
\renewcommand{\cite}[1]{({\it \citenum{#1}})}
\newcommand{\trbcite}[1]{\citeauthor{#1} ({\it \citenum{#1}})}
\setcitestyle{round}

% Reduce spacing between bibliographic items
\setlength{\bibsep}{0pt plus 0.3ex}


% LINE NUMBERING
%------------------------------------------
% TRB likes line numbers on drafts to help reviewers refer to parts of the
% document. Comment out for final versions.

% Use this for draft submissions - line numbering... sort of.
%\usepackage{parano}
%\renewcommand\theparano{\textbf{\scriptsize [\arabic{parano}]}}
% Use this for final submissions - no line numbering
\newcommand{\parano}{}

% Add line numbers
\usepackage[pagewise]{lineno}
  \renewcommand\linenumberfont{\normalfont\small}


% COUNTERS
%------------------------------------------
% TRB requires the total number of words, figures, and tables to be displayed on
% the title page. This is possible under the totcount and the xparse packages on CTAN.
%
% Note that a total word count is added in V 2.1 to print a single value that is 
% calculated as: numberofwords + numberoffigures*250 + numberoftable*250

\newread\somefile
\usepackage{xparse}

% Total world count solution from Tex.SX: http://tex.stackexchange.com/questions/255940/print-a-total-word-count-that-sums-up-the-number-of-words-tables-and-figures
\newcounter{totalwordcounter}
\newcounter{wordcounter}
\makeatletter

\NewDocumentCommand{\wordcount}{s}{%
  \immediate\write18{texcount -sum -1 \jobname.tex > count.txt}%
  \immediate\openin\somefile=count.txt%
  \read\somefile to \@@localdummy%
  \immediate\closein\somefile%
  \setcounter{wordcounter}{\@@localdummy}%
  \IfBooleanF{#1}{%
  \@@localdummy%   print only if not starred version
  }%
}
\makeatother

\usepackage{totcount}
	\regtotcounter{table} 	%count tables
	\regtotcounter{figure} 	%count figures

\newcommand{\wordfigure}{250} % number of words per figure
\newcommand{\wordtable}{250} % number of words per table

\newcommand{\totalwordcount}{%
  \wordcount*% The star allows just getting the number, not printing it
  \setcounter{totalwordcounter}{\value{wordcounter}}%
  \addtocounter{totalwordcounter}{\numexpr\wordfigure*\totvalue{figure}}%
  \addtocounter{totalwordcounter}{\numexpr\wordtable*\totvalue{table}} % 
  \number\value{totalwordcounter}% Output the number
  \renewcommand{\totalwordcount}{\number\value{totalwordcounter}}% Prevent the call again, otherwise the figure/table counter would be added again. 
}

% OTHER PACKAGES
%------------------------------------------
% Add any additional \usepackage declarations here.

\usepackage{graphicx}
\usepackage{subfigure}
\usepackage{setspace}
\usepackage{multicol}
\usepackage{enumitem}
\usepackage{longtable}
\usepackage{multirow}
\usepackage{bm}
\usepackage{mathtools}
\usepackage{amsfonts}
\usepackage{longtable}
\usepackage{amssymb}
\usepackage{amsmath,amsthm}
\graphicspath{{figures/}}
\usepackage{color}\newcommand{\comment}[1]{\textcolor[rgb]{1,0,0}{#1}}
\newcommand{\tabincell}[2]{\begin{tabular}{@{}#1@{}}#2\end{tabular}}
%\renewcommand\arraystretch{1.3}
% TITLEPAGE
%-----------------------------------------
%\usepackage{Sweave}
\begin{document}
%\input{trb_template-concordance}
%\input{trb_template-concordance}
%\input{trb_template-concordance}
	
\thispagestyle{empty}

\pagewiselinenumbers % comment out for final manuscript; comment out if no line numbers on title page

\begin{titlepage}
\begin{flushleft}
% Title
{\MakeUppercase{\bfseries Passenger oriented timetable optimization considering time-dependent demand}}\\[36pt]

{\bfseries Fei Yan} \\
PhD Researcher\\
Department of Transport and Planning\\
Faculty of Civil Engineering and Geosciences\\
Stevinweg 1, 2628 CN Delft, The Netherlands\\
Phone: +31 (15) 27 84914\\
E-mail:{f.yan@tudelft.nl} \\[12pt]

{\bfseries Rob M.P.Goverde}\\
Associate Professor\\
Department of Transport and Planning\\
Faculty of Civil Engineering and Geosciences\\
Stevinweg 1, 2628 CN Delft, The Netherlands\\
Phone: +31 (15) 27 83178\\
E-mail:{r.m.p.goverde@tudelft.nl} \\[12pt]

%% \totalcount (need to have Perl interpreter [e.g. ActivePerl] and enable shell escape in pdfLaTeX, see some examples below)
%http://www.activestate.com/activeperl/downloads
%http://stackoverflow.com/questions/3252957/how-to-execute-shell-script-from-latex
%http://tex.stackexchange.com/questions/233511/inkscape-and-shell-escape-with-texstudio
%Word Count: \wordcount words + \total{figure} figure(s) + \total{table} table(s) = \totalwordcount~words\\[12pt] 
%Command out the above line and manually fill in counts using the line below if you could not get the autocount work.
Word Count: xxxx words + \total{figure} figure(s) + \total{table} table(s) = xxxx~words \comment{No more than 7500}\\[12pt]

Submission Date: \today
\end{flushleft}
\end{titlepage}

%\pagewiselinenumbers % comment out for final manuscript

\newpage
\section{Abstract}
%% ***   Put your Abstract here.   ***
(At most 250 words.)

%\textit{Keywords}:Time-dependent, timetable optimization, PESP,

\newpage
\section{Introduction}
Motivation\\
Literature review\\
framework
% Citations: The \trbcite{TRBGuide} has unique and somewhat 

\section{Problem description}
(use Newell graph)
\begin{figure}[htbp]
	\centering
	\includegraphics[width=0.8\textwidth]{Event_Activity_graph.pdf}
	\caption{EAN graph}
	\label{fig:1}
\end{figure}

\newpage
\section{Model formulation}
Cycle time is $T$.\\ 
Line set is $L$.\\ 
Train set is $\varGamma$\\
$E_{odt}$ is the set of $(o,d,t)$. $(o,d,t)$ represents a arriving event with passenger from station $o$ to $d$ at time $t$. \\
$E$ is the set of all event of train lines, including arrival event $E_{arr}$, departure events $E_{dep}$, through events $E_{thr}$.\\
$A$ is all activities of train lines, $a \in A$.\\
$A_{wait}$ stands for the set of activities from passenger arrival event to train departure event.\\
$\phi_i^a$ is an binary parameter. It represents whether a train with process $a$ could be an alternative choice for passengers at event $i$. It equals one if includes, otherwise zero.

\begin{equation}\label{eq2}
\mathrm{Minimize} \sum_{ i\in E_{odt}} \sum_{\tau \in \varGamma} q_i\cdot y_i^{\tau} \cdot (\pi_j-\pi_i+p_{ij}\cdot T)
\end{equation}
Subject to
\begin{align}
& l_{ij}\leq \pi_j-\pi_i+ p_{ij}\cdot T \leq u_{ij} && \forall (i,j) \in A \label{eq3}\\
& 0\leq \pi_i < T && \forall i \in E \cup E_{odt} \label{eq4}\\
& \pi_i\leq \pi_j+p_{ij}\cdot T+M\cdot(1-y_i^\tau) &&\forall i \in E_{odt}, j \in E_{dep}\\
& \sum_{\tau \in \varGamma} y_i^{\tau}=1 && \forall i \in E_{odt}\\
& \sum_{a\in A} \sum_{i\in E_{odt}} q_i \cdot y_i^{\tau} \cdot \phi_i^a \leq C && \forall \tau \in \varGamma\\
& p_{ij} \in \{0,1\}&& \forall (i,j) \in A \cup A_{wait} \label{eq5}\\
&y_i^{\tau} \in \{0,1\}&& \forall i \in E_{odt}, \tau \in \varGamma \label{eq5}
\end{align}
Simplification:\\
For line $l$ does not have stops on both station $o$ and $d$, the corresponding train $\tau$ will have $y_i^{\tau}=0$.\\
Objective function Linearization:
\begin{equation}\label{eq2}
\mathrm{Minimize} \sum_{ i\in E_{odt}} \sum_{\tau \in \varGamma} q_i\cdot z_i^{\tau}
\end{equation}
Where
\begin{align}
&z_i^{\tau}= y_i^{\tau} \cdot(\pi_j-\pi_i+p_{ij}\cdot T) && \forall i \in E_{odt}, j \in E_{dep}, \tau \in \varGamma
\end{align}
and following constraints are designed to make the linear inequality have the same effect as the upper equation.

\begin{align}
&z_i^{\tau}+M \cdot (1-y_i^{\tau})\geq (\pi_j-\pi_i+p_{ij}\cdot T) && \forall i \in E_{odt}, j \in E_{dep}, \tau \in \varGamma\\
&z_i^{\tau}\in \mathbb{N} && \forall i \in E_{odt}, \tau \in \varGamma
\end{align}

\section{Numerical Experiments}
We could use intercity corridor from Shanghai to Nanjing, but simplify to 10 or 12 stations. And assume a line plan with given stop pattern and corresponding frequencies.

\section{Conclusion}


%\section*{Acknowledgement}
% This work was partially supported by the China Scholarship Council CSC (No.201309110101).
\newpage

\bibliographystyle{trb}
\bibliography{Reference}

% End line numbering
\nolinenumbers
\end{document}
